%% header.tex
%%
%% Copyright (C) 2016 - 2017  Dirk Eddelbuettel
%%
%% This file is part of samples-rmarkdown-metropolis repository.
%%
%% samples-rmarkdown-metropolis is free software: you can redistribute it
%% and/or modify it under the terms of the GNU General Public License as
%% published by the Free Software Foundation, either version 2 of the
%% License, or (at your option) any later version.
%%
%% samples-rmarkdown-metropolis is distributed in the hope that it will be
%% useful, but WITHOUT ANY WARRANTY; without even the implied warranty of
%% MERCHANTABILITY or FITNESS FOR A PARTICULAR PURPOSE.  See the GNU General
%% Public License for more details.
%%
%% You should have received a copy of the GNU General Public License along with
%% samples-rmarkdown-metropolis.  If not, see <http://www.gnu.org/licenses/>.

%% If you have the Fira font installed, to actually have it used it 
%% via rmarkdown you need to declare it here 
%\setsansfont[ItalicFont={Fira Sans Light Italic},BoldFont={Fira Sans},BoldItalicFont={Fira Sans Italic}]{Fira Sans Light}
%\setmonofont[BoldFont={Fira Mono Medium}]{Fira Mono}

%% You can set various Metropolis options via \metroset{} here
%\metroset{....}

%% You can redefine colours, mostly by borrowing from Beamer
\setbeamercolor{frametitle}{bg=gray}
\usepackage{graphicx}
%\graphicspath{ {/home/sahir/git_repositories/math697/images/} }
\usepackage{hyperref, url}
\usepackage{caption}
\usepackage[round,sort]{natbib}   % bibliography omit 'round' option if you prefer square brackets
\usepackage{float}
\usepackage{color, colortbl, xcolor}
\definecolor{lightgray}{RGB}{200,200,200}
\usepackage{array}
\newcolumntype{L}{>{\centering\arraybackslash}m{3cm}} % used for text wrapping in ctable
\usepackage{ctable}
\usepackage{nicefrac}
%\usepackage{calrsfs}
%\DeclareMathAlphabet{\pazocal}{OMS}{zplm}{m}{n}
%\newcommand{\La}{\mathcal{L}}
%\newcommand{\Lb}{\pazocal{L}}
\usepackage{pifont}% http://ctan.org/pkg/pifont
\newcommand{\cmark}{\ding{51}}%
\newcommand{\xmark}{\ding{55}}%
\def\widebar#1{\overline{#1}}

\def\Xmean{\skew3\widebar{X}}
\def\Ymean{\widebar{Y}}
\def\xmean{\bar{x}}
\def\ymean{\bar{y}}
\def\dint{\displaystyle\int}
\def\dsum{\displaystyle\sum}

\newcommand{\xbar}{\bar{x}}
\newcommand{\Xbar}{\bar{X}}
\newcommand{\sumn}{\dsum_{i=1}^{n}}

\metroset{numbering=fraction, block = fill}

%% change fontsize of R code
\let\oldShaded\Shaded
\let\endoldShaded\endShaded
\renewenvironment{Shaded}{\scriptsize\oldShaded}{\endoldShaded}

%% change fontsize of output
\let\oldverbatim\verbatim
\let\endoldverbatim\endverbatim
\renewenvironment{verbatim}{\scriptsize\oldverbatim}{\endoldverbatim}

\newtheorem{proposition}[theorem]{Proposition}
\newtheorem{exercise}[theorem]{Exercise}


%% You also use hyperref, and pick colors 
\hypersetup{colorlinks,citecolor=orange,filecolor=red,linkcolor=brown,urlcolor=blue}

\newcommand {\framedgraphiccaption}[2] {
            \begin{figure}
            \centering
            \includegraphics[width=\textwidth,height=0.8\textheight,keepaspectratio]{#1}
            \caption{#2}
            \end{figure}
}

\newcommand {\framedgraphic}[1] {
            \begin{figure}
            \centering
            \includegraphics[width=\textwidth,height=0.7\textheight,keepaspectratio]{#1}
            \end{figure}
}

%\usepackage{amsthm}
\setbeamertemplate{theorems}[numbered]

%% when rendered with rmarkdown, somehow the unicode char for the dot
%% disappears so we redefine it here -- that is an older comments, seems font-specific
%\renewcommand{\textbullet}{$\cdot$}
%\renewcommand{\itemBullet}{▸}   % unicode U+25b8 'black right pointing small triangle'

%% The institute macro puts a small line for affiliation at the bottom
\institute{McGill University} 

%% We can also place a logo
%\titlegraphic{\hfill\includegraphics[height=1cm]{someLogo.pdf}}

%%% Local Variables:
%%% mode: latex
%%% TeX-master: t
%%% End:
